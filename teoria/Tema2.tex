% Options for packages loaded elsewhere
\PassOptionsToPackage{unicode}{hyperref}
\PassOptionsToPackage{hyphens}{url}
%
\documentclass[
  ignorenonframetext,
]{beamer}
\usepackage{pgfpages}
\setbeamertemplate{caption}[numbered]
\setbeamertemplate{caption label separator}{: }
\setbeamercolor{caption name}{fg=normal text.fg}
\beamertemplatenavigationsymbolsempty
% Prevent slide breaks in the middle of a paragraph
\widowpenalties 1 10000
\raggedbottom
\setbeamertemplate{part page}{
  \centering
  \begin{beamercolorbox}[sep=16pt,center]{part title}
    \usebeamerfont{part title}\insertpart\par
  \end{beamercolorbox}
}
\setbeamertemplate{section page}{
  \centering
  \begin{beamercolorbox}[sep=12pt,center]{part title}
    \usebeamerfont{section title}\insertsection\par
  \end{beamercolorbox}
}
\setbeamertemplate{subsection page}{
  \centering
  \begin{beamercolorbox}[sep=8pt,center]{part title}
    \usebeamerfont{subsection title}\insertsubsection\par
  \end{beamercolorbox}
}
\AtBeginPart{
  \frame{\partpage}
}
\AtBeginSection{
  \ifbibliography
  \else
    \frame{\sectionpage}
  \fi
}
\AtBeginSubsection{
  \frame{\subsectionpage}
}
\usepackage{amsmath,amssymb}
\usepackage{iftex}
\ifPDFTeX
  \usepackage[T1]{fontenc}
  \usepackage[utf8]{inputenc}
  \usepackage{textcomp} % provide euro and other symbols
\else % if luatex or xetex
  \usepackage{unicode-math} % this also loads fontspec
  \defaultfontfeatures{Scale=MatchLowercase}
  \defaultfontfeatures[\rmfamily]{Ligatures=TeX,Scale=1}
\fi
\usepackage{lmodern}
\ifPDFTeX\else
  % xetex/luatex font selection
\fi
% Use upquote if available, for straight quotes in verbatim environments
\IfFileExists{upquote.sty}{\usepackage{upquote}}{}
\IfFileExists{microtype.sty}{% use microtype if available
  \usepackage[]{microtype}
  \UseMicrotypeSet[protrusion]{basicmath} % disable protrusion for tt fonts
}{}
\makeatletter
\@ifundefined{KOMAClassName}{% if non-KOMA class
  \IfFileExists{parskip.sty}{%
    \usepackage{parskip}
  }{% else
    \setlength{\parindent}{0pt}
    \setlength{\parskip}{6pt plus 2pt minus 1pt}}
}{% if KOMA class
  \KOMAoptions{parskip=half}}
\makeatother
\usepackage{xcolor}
\newif\ifbibliography
\usepackage{color}
\usepackage{fancyvrb}
\newcommand{\VerbBar}{|}
\newcommand{\VERB}{\Verb[commandchars=\\\{\}]}
\DefineVerbatimEnvironment{Highlighting}{Verbatim}{commandchars=\\\{\}}
% Add ',fontsize=\small' for more characters per line
\usepackage{framed}
\definecolor{shadecolor}{RGB}{248,248,248}
\newenvironment{Shaded}{\begin{snugshade}}{\end{snugshade}}
\newcommand{\AlertTok}[1]{\textcolor[rgb]{0.94,0.16,0.16}{#1}}
\newcommand{\AnnotationTok}[1]{\textcolor[rgb]{0.56,0.35,0.01}{\textbf{\textit{#1}}}}
\newcommand{\AttributeTok}[1]{\textcolor[rgb]{0.13,0.29,0.53}{#1}}
\newcommand{\BaseNTok}[1]{\textcolor[rgb]{0.00,0.00,0.81}{#1}}
\newcommand{\BuiltInTok}[1]{#1}
\newcommand{\CharTok}[1]{\textcolor[rgb]{0.31,0.60,0.02}{#1}}
\newcommand{\CommentTok}[1]{\textcolor[rgb]{0.56,0.35,0.01}{\textit{#1}}}
\newcommand{\CommentVarTok}[1]{\textcolor[rgb]{0.56,0.35,0.01}{\textbf{\textit{#1}}}}
\newcommand{\ConstantTok}[1]{\textcolor[rgb]{0.56,0.35,0.01}{#1}}
\newcommand{\ControlFlowTok}[1]{\textcolor[rgb]{0.13,0.29,0.53}{\textbf{#1}}}
\newcommand{\DataTypeTok}[1]{\textcolor[rgb]{0.13,0.29,0.53}{#1}}
\newcommand{\DecValTok}[1]{\textcolor[rgb]{0.00,0.00,0.81}{#1}}
\newcommand{\DocumentationTok}[1]{\textcolor[rgb]{0.56,0.35,0.01}{\textbf{\textit{#1}}}}
\newcommand{\ErrorTok}[1]{\textcolor[rgb]{0.64,0.00,0.00}{\textbf{#1}}}
\newcommand{\ExtensionTok}[1]{#1}
\newcommand{\FloatTok}[1]{\textcolor[rgb]{0.00,0.00,0.81}{#1}}
\newcommand{\FunctionTok}[1]{\textcolor[rgb]{0.13,0.29,0.53}{\textbf{#1}}}
\newcommand{\ImportTok}[1]{#1}
\newcommand{\InformationTok}[1]{\textcolor[rgb]{0.56,0.35,0.01}{\textbf{\textit{#1}}}}
\newcommand{\KeywordTok}[1]{\textcolor[rgb]{0.13,0.29,0.53}{\textbf{#1}}}
\newcommand{\NormalTok}[1]{#1}
\newcommand{\OperatorTok}[1]{\textcolor[rgb]{0.81,0.36,0.00}{\textbf{#1}}}
\newcommand{\OtherTok}[1]{\textcolor[rgb]{0.56,0.35,0.01}{#1}}
\newcommand{\PreprocessorTok}[1]{\textcolor[rgb]{0.56,0.35,0.01}{\textit{#1}}}
\newcommand{\RegionMarkerTok}[1]{#1}
\newcommand{\SpecialCharTok}[1]{\textcolor[rgb]{0.81,0.36,0.00}{\textbf{#1}}}
\newcommand{\SpecialStringTok}[1]{\textcolor[rgb]{0.31,0.60,0.02}{#1}}
\newcommand{\StringTok}[1]{\textcolor[rgb]{0.31,0.60,0.02}{#1}}
\newcommand{\VariableTok}[1]{\textcolor[rgb]{0.00,0.00,0.00}{#1}}
\newcommand{\VerbatimStringTok}[1]{\textcolor[rgb]{0.31,0.60,0.02}{#1}}
\newcommand{\WarningTok}[1]{\textcolor[rgb]{0.56,0.35,0.01}{\textbf{\textit{#1}}}}
\usepackage{longtable,booktabs,array}
\usepackage{calc} % for calculating minipage widths
\usepackage{caption}
% Make caption package work with longtable
\makeatletter
\def\fnum@table{\tablename~\thetable}
\makeatother
\setlength{\emergencystretch}{3em} % prevent overfull lines
\providecommand{\tightlist}{%
  \setlength{\itemsep}{0pt}\setlength{\parskip}{0pt}}
\setcounter{secnumdepth}{-\maxdimen} % remove section numbering
\ifLuaTeX
  \usepackage{selnolig}  % disable illegal ligatures
\fi
\usepackage{bookmark}
\IfFileExists{xurl.sty}{\usepackage{xurl}}{} % add URL line breaks if available
\urlstyle{same}
\hypersetup{
  pdftitle={Tema 2 - Documentación con R Markdown},
  pdfauthor={Juan Gabriel Gomila \& María Santos},
  hidelinks,
  pdfcreator={LaTeX via pandoc}}

\title{Tema 2 - Documentación con R Markdown}
\author{Juan Gabriel Gomila \& María Santos}
\date{}

\begin{document}
\frame{\titlepage}

\section{Introducción}\label{introducciuxf3n}

\begin{frame}{Markdown}
\phantomsection\label{markdown}
R Markdown. Es un tipo de fichero en el cual podemos intercalar sin
problema alguno texto, código y fórmulas matemáticas.

Para la mayor parte de las necesidades de este curso, en lo referente a
la creación y composición de este tipo de ficheros, el documento
\emph{\href{https://en.support.wordpress.com/markdown-quick-reference/}{Markdown
Quick Reference}} y la
\href{http://shiny.rstudio.com/images/rm-cheatsheet.pdf.zip.}{chuleta}
de R Markdown deberían ser suficientes.

Sin embargo, a lo largo de este curso iremos ampliando estos contenidos
en algunos temas cuando lo creamos necesario.

Nosotros, en este tema, veremos cómo controlar el comportamiento de los
bloques de código (chunks) al compilar el fichero R Markdown y cómo
escribir fórmulas matemáticas bien formateadas.
\end{frame}

\section{Fórmulas matemáticas}\label{fuxf3rmulas-matemuxe1ticas}

\begin{frame}[fragile]{Cómo escribir}
\phantomsection\label{cuxf3mo-escribir}
Para escribir fórmulas matemáticas bien formateadas utilizaremos la
sintaxis \(\LaTeX\)

\begin{itemize}
\tightlist
\item
  Para tener ecuaciones o fórmulas en el mismo párrafo, escribimos
  nuestro código entre dos símbolos de dólar:
  \texttt{\$}código\texttt{\$}
\item
  Si queremos tener ecuaciones o fórmulas centradas en un párrafo
  aparte, escribimos nuestro código entre dos dobles símbolos de dólar:
  \texttt{\$\$}código\texttt{\$\$}
\end{itemize}

¡Cuidado! Al escribir una fórmula de la forma indicada anteriormente o
simplemente texto en R Markdown, los espacios en blanco son
completamente ignorados. RStudio solamente añade los espacios en blanco
a partir del significado lógico de sus elementos.
\end{frame}

\begin{frame}[fragile]{Espacios en blanco}
\phantomsection\label{espacios-en-blanco}
\textbf{Ejemplo}

Para que veáis que RStudio ignora el exceso de espacios en blanco, aquí
os damos un ejemplo en el que hemos introducido espacios innecesarios:

Código: \texttt{En\ en\ instituto} \(\ \ \ \ \ \)
\texttt{nos\ enseñaron\ que\ las\ raíces\ de\ las\ ecuaciones\ de\ tercer\ grado,\ de\ la\ forma\ \$Ax\^{}3+Bx\^{}2+Cx+D=0\$,\ se\ encuentran\ mediante\ \textbackslash{}textit\{la\ Regla\ de\ Ruffini\}.\ Por\ su\ parte,}\(\ \ \ \ \ \ \ \ \ \)
\texttt{las\ raíces\ de\ las\ ecuaciones\ de\ segundo\ grado\ de\ la\ forma\ \$\textbackslash{}alpha\ x\^{}2+\textbackslash{}beta\ x+\textbackslash{}gamma=0\$\ se\ encuentran\ siguiendo\ la\ fórmula\ \$\$x\ =\ \textbackslash{}frac\{-\textbackslash{}beta\textbackslash{}pm\textbackslash{}sqrt\{\textbackslash{}beta\^{}2}
\(\ \ \ \ \ \ \ \ \ \ \ \ \)\texttt{-4\textbackslash{}alpha\textbackslash{}gamma\}\}\{2\textbackslash{}alpha\}\$\$.}

Resultado: En en instituto nos enseñaron que las raíces de las
ecuaciones de tercer grado, de la forma \(Ax^3+Bx^2+Cx+D=0\), se
encuentran mediante \emph{la Regla de Ruffini}. Por su parte, las raíces
de las ecuaciones de segundo grado de la forma
\(\alpha x^2+\beta x+\gamma=0\) se encuentran siguiendo la fórmula
\[x = \frac{-\beta\pm\sqrt{\beta^2    -4\alpha\gamma}}{2\alpha}\].
\end{frame}

\begin{frame}{Símbolos}
\phantomsection\label{suxedmbolos}
Hay muchísimos símbolos matemáticos que puedes escribirse con la
sintaxis \(\LaTeX\). En el ejemplo anterior ya os hemos mostrado unos
pocos. En este tema, nosotros solo veremos los más utilizados.

Para quien quiera ir más allá, aquí os dejamos un
\href{http://www.ptep-online.com/ctan/symbols.pdf}{documento muy útil}
con gran cantidad de símbolos de \(\LaTeX\).
\end{frame}

\begin{frame}[fragile]{Símbolos matemáticos - Básico}
\phantomsection\label{suxedmbolos-matemuxe1ticos---buxe1sico}
\begin{longtable}[]{@{}lll@{}}
\toprule\noalign{}
Significado & Código & Resultado \\
\midrule\noalign{}
\endhead
Suma & \texttt{+} & \(+\) \\
Resta & \texttt{-} & \(-\) \\
Producto & \texttt{\textbackslash{}cdot} & \(\cdot\) \\
Producto & \texttt{\textbackslash{}times} & \(\times\) \\
División & \texttt{\textbackslash{}div} & \(\div\) \\
Potencia & \texttt{a\^{}\{x\}} & \(a^{x}\) \\
Subíndice & \texttt{a\_\{i\}} & \(a_{i}\) \\
\bottomrule\noalign{}
\end{longtable}
\end{frame}

\begin{frame}[fragile]{Símbolos matemáticos - Básico}
\phantomsection\label{suxedmbolos-matemuxe1ticos---buxe1sico-1}
\begin{longtable}[]{@{}lll@{}}
\toprule\noalign{}
Significado & Código & Resultado \\
\midrule\noalign{}
\endhead
Fracción & \texttt{\textbackslash{}frac\{a\}\{b\}} & \(\frac{a}{b}\) \\
Más menos & \texttt{\textbackslash{}pm} & \(\pm\) \\
Raíz n-ésima & \texttt{\textbackslash{}sqrt{[}n{]}\{x\}} &
\(\sqrt[n]{x}\) \\
Unión & \texttt{cup} & \(\cup\) \\
Intersección & \texttt{\textbackslash{}cap} & \(\cap\) \\
OR lógico & \texttt{\textbackslash{}vee} & \(\vee\) \\
AND lógico & \texttt{\textbackslash{}wedge} & \(\wedge\) \\
\bottomrule\noalign{}
\end{longtable}
\end{frame}

\begin{frame}[fragile]{Símbolos matemáticos - Relaciones}
\phantomsection\label{suxedmbolos-matemuxe1ticos---relaciones}
\begin{longtable}[]{@{}lll@{}}
\toprule\noalign{}
Significado & Código & Resultado \\
\midrule\noalign{}
\endhead
Igual & \texttt{=} & \(=\) \\
Aproximado & \texttt{\textbackslash{}approx} & \(\approx\) \\
No igual & \texttt{\textbackslash{}ne} & \(\ne\) \\
Mayor que & \texttt{\textgreater{}} & \(>\) \\
Menor que & \texttt{\textless{}} & \(<\) \\
Mayor o igual que & \texttt{\textbackslash{}ge} & \(\ge\) \\
Menor o igual que & \texttt{\textbackslash{}le} & \(\le\) \\
\bottomrule\noalign{}
\end{longtable}
\end{frame}

\begin{frame}[fragile]{Símbolos matemáticos - Operadores}
\phantomsection\label{suxedmbolos-matemuxe1ticos---operadores}
\begin{longtable}[]{@{}lll@{}}
\toprule\noalign{}
Significado & Código & Resultado \\
\midrule\noalign{}
\endhead
Sumatorio & \texttt{\textbackslash{}sum\_\{i=0\}\^{}\{n\}} &
\(\sum_{i=0}^{n}\) \\
Productorio & \texttt{\textbackslash{}prod\_\{i=0\}\^{}\{n\}} &
\(\prod_{i=0}^{n}\) \\
Integral & \texttt{\textbackslash{}int\_\{a\}\^{}\{b\}} &
\(\int_{a}^{b}\) \\
Unión (grande) & \texttt{\textbackslash{}bigcup} & \(\bigcup\) \\
Intersección (grande) & \texttt{\textbackslash{}bigcap} & \(\bigcap\) \\
OR lógico (grande) & \texttt{\textbackslash{}bigvee} & \(\bigvee\) \\
AND lógico (grande) & \texttt{\textbackslash{}bigwedge} &
\(\bigwedge\) \\
\bottomrule\noalign{}
\end{longtable}
\end{frame}

\begin{frame}[fragile]{Símbolos matemáticos - Delimitadores}
\phantomsection\label{suxedmbolos-matemuxe1ticos---delimitadores}
\begin{longtable}[]{@{}lll@{}}
\toprule\noalign{}
Significado & Código & Resultado \\
\midrule\noalign{}
\endhead
Paréntesis & \texttt{()} & \((\ )\) \\
Corchetes & \texttt{{[}{]}} & \([\ ]\) \\
Llaves & \texttt{\textbackslash{}\{\ \textbackslash{}\}} & \(\{\ \}\) \\
Diamante & \texttt{\textbackslash{}langle\ \textbackslash{}rangle} &
\(\langle\ \rangle\) \\
Parte entera por defecto &
\texttt{\textbackslash{}lfloor\ \textbackslash{}rfloor} &
\(\lfloor\  \rfloor\) \\
Parte entera por exceso &
\texttt{\textbackslash{}lceil\ \textbackslash{}rceil} &
\(\lceil\ \rceil\) \\
Espacio en blanco & \texttt{hola\textbackslash{}\ caracola} &
\(hola\ caracola\) \\
\bottomrule\noalign{}
\end{longtable}
\end{frame}

\begin{frame}[fragile]{Símbolos matemáticos - Letras griegas}
\phantomsection\label{suxedmbolos-matemuxe1ticos---letras-griegas}
\begin{longtable}[]{@{}lll@{}}
\toprule\noalign{}
Significado & Código & Resultado \\
\midrule\noalign{}
\endhead
Alpha & \texttt{\textbackslash{}alpha} & \(\alpha\) \\
Beta & \texttt{\textbackslash{}beta} & \(\beta\) \\
Gamma & \texttt{\textbackslash{}gamma\ \textbackslash{}Gamma} &
\(\gamma\  \Gamma\) \\
Delta & \texttt{\textbackslash{}delta\ \textbackslash{}Delta} &
\(\delta\  \Delta\) \\
Epsilon & \texttt{\textbackslash{}epsilon} & \(\epsilon\) \\
Epsilon & \texttt{\textbackslash{}varepsilon} & \(\varepsilon\) \\
Zeta & \texttt{\textbackslash{}zeta} & \(\zeta\) \\
\bottomrule\noalign{}
\end{longtable}
\end{frame}

\begin{frame}[fragile]{Símbolos matemáticos - Letras griegas}
\phantomsection\label{suxedmbolos-matemuxe1ticos---letras-griegas-1}
\begin{longtable}[]{@{}lll@{}}
\toprule\noalign{}
Significado & Código & Resultado \\
\midrule\noalign{}
\endhead
Eta & \texttt{\textbackslash{}eta} & \(\eta\) \\
Theta & \texttt{\textbackslash{}theta\ \textbackslash{}Theta} &
\(\theta\ \Theta\) \\
Kappa & \texttt{\textbackslash{}kappa} & \(\kappa\) \\
Lambda & \texttt{\textbackslash{}lambda\ \textbackslash{}Lambda} &
\(\lambda\  \Lambda\) \\
Mu & \texttt{\textbackslash{}mu} & \(\mu\) \\
Nu & \texttt{\textbackslash{}nu} & \(\nu\) \\
Xi & \texttt{\textbackslash{}xi\ \textbackslash{}Xi} & \(\xi\ \Xi\) \\
\bottomrule\noalign{}
\end{longtable}
\end{frame}

\begin{frame}[fragile]{Símbolos matemáticos - Letras griegas}
\phantomsection\label{suxedmbolos-matemuxe1ticos---letras-griegas-2}
\begin{longtable}[]{@{}lll@{}}
\toprule\noalign{}
Significado & Código & Resultado \\
\midrule\noalign{}
\endhead
Pi & \texttt{\textbackslash{}pi\ \textbackslash{}Pi} & \(\pi\ \Pi\) \\
Rho & \texttt{\textbackslash{}rho} & \(\rho\) \\
Sigma & \texttt{\textbackslash{}sigma\ \textbackslash{}Sigma} &
\(\sigma\ \Sigma\) \\
Tau & \texttt{\textbackslash{}tau} & \(\tau\) \\
Upsilon & \texttt{\textbackslash{}upsilon\ \textbackslash{}Upsilon} &
\(\upsilon\ \Upsilon\) \\
Phi & \texttt{\textbackslash{}phi\ \textbackslash{}Phi} &
\(\phi\ \Phi\) \\
Phi & \texttt{\textbackslash{}varphi} & \(\varphi\) \\
\bottomrule\noalign{}
\end{longtable}
\end{frame}

\begin{frame}[fragile]{Símbolos matemáticos - Letras griegas}
\phantomsection\label{suxedmbolos-matemuxe1ticos---letras-griegas-3}
\begin{longtable}[]{@{}lll@{}}
\toprule\noalign{}
Significado & Código & Resultado \\
\midrule\noalign{}
\endhead
Chi & \texttt{\textbackslash{}chi} & \(\chi\) \\
Psi & \texttt{\textbackslash{}psi\ \textbackslash{}Psi} &
\(\psi\ \Psi\) \\
Omega & \texttt{\textbackslash{}omega\ \textbackslash{}Omega} &
\(\omega\ \Omega\) \\
\bottomrule\noalign{}
\end{longtable}
\end{frame}

\begin{frame}[fragile]{Símbolos matemáticos - Acentos matemáticos}
\phantomsection\label{suxedmbolos-matemuxe1ticos---acentos-matemuxe1ticos}
\begin{longtable}[]{@{}lll@{}}
\toprule\noalign{}
Significado & Código & Resultado \\
\midrule\noalign{}
\endhead
Gorrito & \texttt{\textbackslash{}hat\{x\}} & \(\hat{x}\) \\
Barra & \texttt{\textbackslash{}bar\{x\}} & \(\bar{x}\) \\
Punto 1 & \texttt{\textbackslash{}dot\{x\}} & \(\dot{x}\) \\
Punto 2 & \texttt{\textbackslash{}ddot\{x\}} & \(\ddot{x}\) \\
Punto 3 & \texttt{\textbackslash{}dddot\{x\}} & \(\dddot{x}\) \\
Tilde & \texttt{\textbackslash{}tilde\{x\}} & \(\tilde{x}\) \\
Vector & \texttt{\textbackslash{}vec\{x\}} & \(\vec{x}\) \\
\bottomrule\noalign{}
\end{longtable}
\end{frame}

\begin{frame}[fragile]{Símbolos matemáticos - Acentos expansibles}
\phantomsection\label{suxedmbolos-matemuxe1ticos---acentos-expansibles}
\begin{longtable}[]{@{}lll@{}}
\toprule\noalign{}
Significado & Código & Resultado \\
\midrule\noalign{}
\endhead
Gorrito & \texttt{\textbackslash{}widehat\{xyz\}} & \(\widehat{xyz}\) \\
Barra & \texttt{\textbackslash{}overline\{xyz\}} & \(\overline{xyz}\) \\
Subrayado & \texttt{\textbackslash{}underline\{xyz\}} &
\(\underline{xyz}\) \\
Llave superior & \texttt{\textbackslash{}overbrace\{xyz\}} &
\(\overbrace{xyz}\) \\
Llave inferior & \texttt{\textbackslash{}underbrace\{xyz\}} &
\(\underbrace{xyz}\) \\
Tilde & \texttt{\textbackslash{}widetilde\{xyz\}} &
\(\widetilde{xyz}\) \\
Vector & \texttt{\textbackslash{}overrightarrow\{xyz\}} &
\(\overrightarrow{xyz}\) \\
\bottomrule\noalign{}
\end{longtable}
\end{frame}

\begin{frame}[fragile]{Símbolos matemáticos - Flechas}
\phantomsection\label{suxedmbolos-matemuxe1ticos---flechas}
\begin{longtable}[]{@{}
  >{\raggedright\arraybackslash}p{(\columnwidth - 4\tabcolsep) * \real{0.3333}}
  >{\raggedright\arraybackslash}p{(\columnwidth - 4\tabcolsep) * \real{0.3333}}
  >{\raggedright\arraybackslash}p{(\columnwidth - 4\tabcolsep) * \real{0.3333}}@{}}
\toprule\noalign{}
\begin{minipage}[b]{\linewidth}\raggedright
Significado
\end{minipage} & \begin{minipage}[b]{\linewidth}\raggedright
Código
\end{minipage} & \begin{minipage}[b]{\linewidth}\raggedright
Resultado
\end{minipage} \\
\midrule\noalign{}
\endhead
Simple & \texttt{\textbackslash{}leftarrow\ \textbackslash{}rightarrow}
& \(\leftarrow\ \rightarrow\) \\
Doble & \texttt{\textbackslash{}Leftarrow\ \textbackslash{}Rightarrow} &
\(\Leftarrow\ \Rightarrow\) \\
Simple larga &
\texttt{\textbackslash{}longleftarrow\ \textbackslash{}longrightarrow} &
\(\longleftarrow\  \longrightarrow\) \\
Doble larga &
\texttt{\textbackslash{}Longleftarrow\ \textbackslash{}Longrightarrow} &
\(\Longleftarrow\ \Longrightarrow\) \\
Doble sentido simple & \texttt{\textbackslash{}leftrightarrow} &
\(\leftrightarrow\) \\
Doble sentido doble & \texttt{\textbackslash{}Leftrightarrow} &
\(\Leftrightarrow\) \\
\bottomrule\noalign{}
\end{longtable}
\end{frame}

\begin{frame}[fragile]{Símbolos matemáticos - Flechas}
\phantomsection\label{suxedmbolos-matemuxe1ticos---flechas-1}
\begin{longtable}[]{@{}
  >{\raggedright\arraybackslash}p{(\columnwidth - 4\tabcolsep) * \real{0.3333}}
  >{\raggedright\arraybackslash}p{(\columnwidth - 4\tabcolsep) * \real{0.3333}}
  >{\raggedright\arraybackslash}p{(\columnwidth - 4\tabcolsep) * \real{0.3333}}@{}}
\toprule\noalign{}
\begin{minipage}[b]{\linewidth}\raggedright
Significado
\end{minipage} & \begin{minipage}[b]{\linewidth}\raggedright
Código
\end{minipage} & \begin{minipage}[b]{\linewidth}\raggedright
Resultado
\end{minipage} \\
\midrule\noalign{}
\endhead
Doble sentido larga simple & \texttt{\textbackslash{}longleftrightarrow}
& \(\longleftrightarrow\) \\
Doble sentido larga doble & \texttt{\textbackslash{}Longleftrightarrow}
& \(\Longleftrightarrow\) \\
Mapea & \texttt{\textbackslash{}mapsto} & \(\mapsto\) \\
Arriba & \texttt{\textbackslash{}uparrow} & \(\uparrow\) \\
Abajo & \texttt{\textbackslash{}downarrow} & \(\downarrow\) \\
\bottomrule\noalign{}
\end{longtable}
\end{frame}

\begin{frame}[fragile]{Símbolos matemáticos - Funciones}
\phantomsection\label{suxedmbolos-matemuxe1ticos---funciones}
\begin{longtable}[]{@{}lll@{}}
\toprule\noalign{}
Significado & Código & Resultado \\
\midrule\noalign{}
\endhead
Seno & \texttt{\textbackslash{}sin} & \(\sin\) \\
Coseno & \texttt{\textbackslash{}cos} & \(\cos\) \\
Tangente & \texttt{\textbackslash{}tan} & \(\tan\) \\
Arcoseno & \texttt{\textbackslash{}arcsin} & \(\arcsin\) \\
Arcocoseno & \texttt{\textbackslash{}arccos} & \(\arccos\) \\
Arcotangente & \texttt{\textbackslash{}arctan} & \(\arctan\) \\
\bottomrule\noalign{}
\end{longtable}
\end{frame}

\begin{frame}[fragile]{Símbolos matemáticos - Funciones}
\phantomsection\label{suxedmbolos-matemuxe1ticos---funciones-1}
\begin{longtable}[]{@{}lll@{}}
\toprule\noalign{}
Significado & Código & Resultado \\
\midrule\noalign{}
\endhead
Exponencial & \texttt{\textbackslash{}exp} & \(\exp\) \\
Logaritmo & \texttt{\textbackslash{}log} & \(\log\) \\
Logaritmo neperiano & \texttt{\textbackslash{}ln} & \(\ln\) \\
Máximo & \texttt{\textbackslash{}max} & \(\max\) \\
Mínimo & \texttt{\textbackslash{}min} & \(\min\) \\
Límite & \texttt{\textbackslash{}lim} & \(\lim\) \\
\bottomrule\noalign{}
\end{longtable}
\end{frame}

\begin{frame}[fragile]{Símbolos matemáticos - Funciones}
\phantomsection\label{suxedmbolos-matemuxe1ticos---funciones-2}
\begin{longtable}[]{@{}lll@{}}
\toprule\noalign{}
Significado & Código & Resultado \\
\midrule\noalign{}
\endhead
Supremo & \texttt{\textbackslash{}sup} & \(\sup\) \\
Ínfimo & \texttt{\textbackslash{}inf} & \(\inf\) \\
Determinante & \texttt{\textbackslash{}det} & \(\det\) \\
Argumento & \texttt{\textbackslash{}arg} & \(\arg\) \\
\bottomrule\noalign{}
\end{longtable}
\end{frame}

\begin{frame}[fragile]{Símbolos matemáticos - Otros}
\phantomsection\label{suxedmbolos-matemuxe1ticos---otros}
\begin{longtable}[]{@{}lll@{}}
\toprule\noalign{}
Significado & Código & Resultado \\
\midrule\noalign{}
\endhead
Puntos suspensivos bajos & \texttt{\textbackslash{}ldots} &
\(\ldots\) \\
Puntos suspensivos centrados & \texttt{\textbackslash{}cdots} &
\(\cdots\) \\
Puntos suspensivos verticales & \texttt{\textbackslash{}vdots} &
\(\vdots\) \\
Puntos suspensivos diagonales & \texttt{\textbackslash{}ddots} &
\(\ddots\) \\
Cuantificador existencial & \texttt{\textbackslash{}exists} &
\(\exists\) \\
Cuantificador universal & \texttt{\textbackslash{}forall} &
\(\forall\) \\
Infinito & \texttt{\textbackslash{}infty} & \(\infty\) \\
\bottomrule\noalign{}
\end{longtable}
\end{frame}

\begin{frame}[fragile]{Símbolos matemáticos - Otros}
\phantomsection\label{suxedmbolos-matemuxe1ticos---otros-1}
\begin{longtable}[]{@{}lll@{}}
\toprule\noalign{}
Significado & Código & Resultado \\
\midrule\noalign{}
\endhead
Aleph & \texttt{\textbackslash{}aleph} & \(\aleph\) \\
Conjunto vacío & \texttt{\textbackslash{}emptyset} & \(\emptyset\) \\
Negación & \texttt{\textbackslash{}neg} & \(\neg\) \\
Barra invertida & \texttt{\textbackslash{}backslash} & \(\backslash\) \\
Dollar & \texttt{\textbackslash{}\$} & \(\$\) \\
Porcentaje & \texttt{\textbackslash{}\%} & \(\%\) \\
Parcial & \texttt{\textbackslash{}partial} & \(\partial\) \\
\bottomrule\noalign{}
\end{longtable}
\end{frame}

\begin{frame}[fragile]{Símbolos matemáticos - Tipos de letra}
\phantomsection\label{suxedmbolos-matemuxe1ticos---tipos-de-letra}
\begin{longtable}[]{@{}lll@{}}
\toprule\noalign{}
Significado & Código & Resultado \\
\midrule\noalign{}
\endhead
Negrita & \texttt{\textbackslash{}mathbf\{palabra\}} &
\(\mathbf{palabra}\) \\
Negrita & \texttt{\textbackslash{}boldsymbol\{palabra\}} &
\(\boldsymbol{palabra}\) \\
Negrita de pizarra & \texttt{\textbackslash{}mathbb\{NZQRC\}} &
\(\mathbb{NZQRC}\) \\
Caligráfica & \texttt{\textbackslash{}mathcal\{NZQRC\}} &
\(\mathcal{NZQRC}\) \\
Gótica & \texttt{\textbackslash{}mathfrak\{NZQRC\}} &
\(\mathfrak{NZQRC}\) \\
\bottomrule\noalign{}
\end{longtable}
\end{frame}

\begin{frame}[fragile]{Observaciones}
\phantomsection\label{observaciones}
\begin{itemize}
\item
  A la hora de componer en el interior de un párrafo una fracción,
  existen dos formas: adaptada al tamaño del
  texto,\texttt{\$\textbackslash{}frac\{a\}\{b\}\$}, que resulta en
  \(\frac{a}{b}\); o a tamaño real,
  \texttt{\$\textbackslash{}dfrac\{a\}\{b\}\$}, que da lugar a
  \(\dfrac{a}{b}\).
\item
  Podemos especificar que los delimitadores se adapten a la altura de la
  expresión que envuelven utilizando \texttt{\textbackslash{}left} y
  \texttt{\textbackslash{}right}. Observad el cambio en el siguiente
  ejemplo: \texttt{\$(\textbackslash{}dfrac\{a\}\{b\})\$} y
  \texttt{\$\textbackslash{}left(\textbackslash{}dfrac\{a\}\{b\}\textbackslash{}right)\$}
  producen, respectivamente \((\dfrac{a}{b})\) y
  \(\left(\dfrac{a}{b}\right)\).
\end{itemize}
\end{frame}

\begin{frame}[fragile]{Matrices}
\phantomsection\label{matrices}
\texttt{\$\$\textbackslash{}begin\{matrix\}\ a\_\{11\}\ \&\ a\_\{12\}\ \&\ a\_\{13\}\textbackslash{}\textbackslash{}\ a\_\{21\}\ \&\ a\_\{22\}\ \&\ a\_\{23\}\ \textbackslash{}end\{matrix\}\$\$}

\[\begin{matrix}
a_{11} & a_{12} & a_{13}\\
a_{21} & a_{22} & a_{23}
\end{matrix}\]

\texttt{\$\$\textbackslash{}begin\{pmatrix\}\ a\_\{11\}\ \&\ a\_\{12\}\ \&\ a\_\{13\}\textbackslash{}\textbackslash{}\ a\_\{21\}\ \&\ a\_\{22\}\ \&\ a\_\{23\}\ \textbackslash{}end\{pmatrix\}\$\$}

\[\begin{pmatrix}
a_{11} & a_{12} & a_{13}\\
a_{21} & a_{22} & a_{23}
\end{pmatrix}\]
\end{frame}

\begin{frame}[fragile]{Matrices}
\phantomsection\label{matrices-1}
\texttt{\$\$\textbackslash{}begin\{vmatrix\}\ a\_\{11\}\ \&\ a\_\{12\}\ \&\ a\_\{13\}\textbackslash{}\textbackslash{}\ a\_\{21\}\ \&\ a\_\{22\}\ \&\ a\_\{23\}\ \textbackslash{}end\{vmatrix\}\$\$}

\[\begin{vmatrix}
a_{11} & a_{12} & a_{13}\\
a_{21} & a_{22} & a_{23}
\end{vmatrix}\]

\texttt{\$\$\textbackslash{}begin\{bmatrix\}\ a\_\{11\}\ \&\ a\_\{12\}\ \&\ a\_\{13\}\textbackslash{}\textbackslash{}\ a\_\{21\}\ \&\ a\_\{22\}\ \&\ a\_\{23\}\ \textbackslash{}end\{bmatrix\}\$\$}

\[\begin{bmatrix}
a_{11} & a_{12} & a_{13}\\
a_{21} & a_{22} & a_{23}
\end{bmatrix}\]
\end{frame}

\begin{frame}[fragile]{Matrices}
\phantomsection\label{matrices-2}
\texttt{\$\$\textbackslash{}begin\{Bmatrix\}\ a\_\{11\}\ \&\ a\_\{12\}\ \&\ a\_\{13\}\textbackslash{}\textbackslash{}\ a\_\{21\}\ \&\ a\_\{22\}\ \&\ a\_\{23\}\ \textbackslash{}end\{Bmatrix\}\$\$}

\[\begin{Bmatrix}
a_{11} & a_{12} & a_{13}\\
a_{21} & a_{22} & a_{23}
\end{Bmatrix}\]

\texttt{\$\$\textbackslash{}begin\{Vmatrix\}\ a\_\{11\}\ \&\ a\_\{12\}\ \&\ a\_\{13\}\textbackslash{}\textbackslash{}\ a\_\{21\}\ \&\ a\_\{22\}\ \&\ a\_\{23\}\ \textbackslash{}end\{Vmatrix\}\$\$}

\[\begin{Vmatrix}
a_{11} & a_{12} & a_{13}\\
a_{21} & a_{22} & a_{23}
\end{Vmatrix}\]
\end{frame}

\begin{frame}[fragile]{Sistema de ecuaciones}
\phantomsection\label{sistema-de-ecuaciones}
\texttt{\textbackslash{}begin\{array\}\{l\}\textbackslash{}end\{array\}}
nos produce una tabla alineada a la izquierda. El hecho de introducir el
código \texttt{\textbackslash{}left.\ \textbackslash{}right.} hace que
el delimitador respectivo no aparezca.

\texttt{\$\$\textbackslash{}left.\textbackslash{}begin\{array\}\{l\}\ ax+by=c\textbackslash{}\textbackslash{}\ ex-fy=g\ \textbackslash{}end\{array\}\textbackslash{}right\textbackslash{}\}\$\$}

\[\left.\begin{array}{l}
ax+by=c\\
ex-fy=g
\end{array}\right\}\]

\texttt{\$\$\textbar{}x\textbar{}=\textbackslash{}left\textbackslash{}\{\textbackslash{}begin\{array\}\{l\}\ -x\ \&\ \textbackslash{}text\{si\ \}x\textbackslash{}le\ 0\textbackslash{}\textbackslash{}\ x\ \&\ \textbackslash{}text\{si\ \}x\textbackslash{}ge\ 0\ \textbackslash{}end\{array\}\textbackslash{}right.\$\$}

\[|x|=\left\{\begin{array}{l}
-x & \text{si }x\le 0\\
x & \text{si }x\ge 0
\end{array}\right.\]

La función \texttt{text\{\}} nos permite introducir texto en fórmulas
matemáticas.
\end{frame}

\section{Parámetros de los chuncks de
R}\label{paruxe1metros-de-los-chuncks-de-r}

\begin{frame}[fragile]{Chunks de R}
\phantomsection\label{chunks-de-r}
Chunk. Bloque de código.

Los bloques de código de R dentro de un documento R Markdown se indican
de la manera siguiente

```\{r\}

\texttt{x\ =\ 1+1}

\texttt{x}

\begin{Shaded}
\begin{Highlighting}[]

\NormalTok{que resulta en}


\NormalTok{\textasciigrave{}\textasciigrave{}\textasciigrave{} r}
\NormalTok{x = 1+1}
\NormalTok{x}
\end{Highlighting}
\end{Shaded}
\end{frame}

\begin{frame}{Chunks de R}
\phantomsection\label{chunks-de-r-1}
Hay diversas opciones de crear un bloque de código de R:

\begin{itemize}
\tightlist
\item
  Ir al menú desplegable de ``Chunks'' y seleccionar el de R
\item
  Introducir manualmente
\item
  Alt + Command + I (para Mac) o Alt + Control + I (para Windows)
\end{itemize}
\end{frame}

\begin{frame}[fragile]{Chunks de R}
\phantomsection\label{chunks-de-r-2}
A los chunks se les puede poner etiqueta, para así localizarlos de
manera más fácil. Por ejemplo

```\{r PrimerChunk\}

\texttt{x\ =\ 1+2+3}

\begin{Shaded}
\begin{Highlighting}[]

\NormalTok{\textbackslash{}n}


\NormalTok{\textless{}div class = "r{-}code"\textgreater{}}
\NormalTok{\textasciigrave{}\textasciigrave{}\textasciigrave{}\{r SegundoChunk\}}

\NormalTok{\textasciigrave{} y = 1*2*3\textasciigrave{}}

\NormalTok{\textasciigrave{}\textasciigrave{}\textasciigrave{}\textless{}/div\textgreater{}}




\NormalTok{\#\# Parámetros de los chunks}

\NormalTok{La parte entre llaves también puede contener diversos parámetros, separados por comas entre ellos y separados de la etiqueta (o de r, si hemos decidido no poner ninguna).}

\NormalTok{Estos parámetros determinan el comportamiento del bloque al compilar el documento pulsando el botón \textasciigrave{}Knit\textasciigrave{} situado en la barra superior del área de trabajo.}

\NormalTok{\#\# Parámetros de los chunks}

\NormalTok{Código |  Significado                                  }
\NormalTok{{-}{-}{-}{-}{-}{-}{-}{-}{-}{-}{-}{-}{-}{-}{-}{-}{-}{-}{-}{-}|{-}{-}{-}{-}{-}{-}{-}{-}{-}{-}{-}{-}{-}{-}{-}{-}{-}{-}{-}{-}}
\NormalTok{\textasciigrave{}echo\textasciigrave{} | Si lo igualamos a \textasciigrave{}TRUE\textasciigrave{}, que es el valor por defecto, estaremos diciendo que queremos que se muestre el código fuente del chunk. En cambio, igualado a \textasciigrave{}FALSE\textasciigrave{}, no se mostrará}
\NormalTok{\textasciigrave{}eval\textasciigrave{} | Si lo igualamos a \textasciigrave{}TRUE\textasciigrave{}, que es el valor por defecto, estaremos diciendo que queremos que se evalúe el código. En cambio, igualado a \textasciigrave{}FALSE\textasciigrave{}, no se evaluará}
\NormalTok{\textasciigrave{}message\textasciigrave{} | Nos permite indicar si queremos que se muestren los mensajes que R produce al ejecutar código. Igualado a \textasciigrave{}TRUE\textasciigrave{} se muestran, igualado a \textasciigrave{}FALSE\textasciigrave{} no}
\NormalTok{\textasciigrave{}warning\textasciigrave{} | Nos permite indicar si queremos que se muestren los mensajes de advertencia que producen algunas funciones al ejecutarse. Igualado a \textasciigrave{}TRUE\textasciigrave{} se muestran, igualado a \textasciigrave{}FALSE\textasciigrave{} no}

\NormalTok{\#\# Parámetros de los chunks}

\NormalTok{\textless{}div class = "r{-}code"\textgreater{}}
\NormalTok{\textasciigrave{}\textasciigrave{}\textasciigrave{}\{r, echo=FALSE\}}

\NormalTok{\textasciigrave{} sec = 10:20\textasciigrave{}}

\NormalTok{\textasciigrave{}sec\textasciigrave{}}

\NormalTok{\textasciigrave{}cumsum(sec)\textasciigrave{}}

\NormalTok{\textasciigrave{}\textasciigrave{}\textasciigrave{}\textless{}/div\textgreater{}}

\NormalTok{\textbackslash{}n}

\NormalTok{No aparece}

\NormalTok{\#\# Parámetros de los chunks}

\NormalTok{\textless{}div class = "r{-}code"\textgreater{}}
\NormalTok{\textasciigrave{}\textasciigrave{}\textasciigrave{}\{r, echo=TRUE, message = TRUE\}}

\NormalTok{\textasciigrave{}library(car)\textasciigrave{}}

\NormalTok{\textasciigrave{}head(cars,3)\textasciigrave{}}

\NormalTok{\textasciigrave{}\textasciigrave{}\textasciigrave{}\textless{}/div\textgreater{}}

\NormalTok{\textbackslash{}n}


\NormalTok{\textasciigrave{}\textasciigrave{}\textasciigrave{} r}
\NormalTok{library(car)}
\end{Highlighting}
\end{Shaded}

\begin{verbatim}
## Cargando paquete requerido: carData
\end{verbatim}

\begin{Shaded}
\begin{Highlighting}[]
\FunctionTok{head}\NormalTok{(cars,}\DecValTok{3}\NormalTok{)}
\end{Highlighting}
\end{Shaded}

\begin{verbatim}
##   speed dist
## 1     4    2
## 2     4   10
## 3     7    4
\end{verbatim}
\end{frame}

\begin{frame}[fragile]{Parámetros de los chunks}
\phantomsection\label{paruxe1metros-de-los-chunks}
```\{r, echo=TRUE, message = FALSE, comment = NA\}

\texttt{library(car)}

\texttt{head(cars,3)}

\begin{Shaded}
\begin{Highlighting}[]

\NormalTok{\textbackslash{}n}


\NormalTok{\textasciigrave{}\textasciigrave{}\textasciigrave{} r}
\NormalTok{library(car)}
\NormalTok{head(cars,3)}
\end{Highlighting}
\end{Shaded}

\begin{verbatim}
  speed dist
1     4    2
2     4   10
3     7    4
\end{verbatim}

Fijaos que \texttt{comment=NA} evita que aparezcan los \texttt{\#\#}
\end{frame}

\begin{frame}[fragile]{Parámetros de los chunks}
\phantomsection\label{paruxe1metros-de-los-chunks-1}
\begin{longtable}[]{@{}
  >{\raggedright\arraybackslash}p{(\columnwidth - 4\tabcolsep) * \real{0.3333}}
  >{\raggedright\arraybackslash}p{(\columnwidth - 4\tabcolsep) * \real{0.3333}}
  >{\raggedright\arraybackslash}p{(\columnwidth - 4\tabcolsep) * \real{0.3333}}@{}}
\toprule\noalign{}
\begin{minipage}[b]{\linewidth}\raggedright
Significado
\end{minipage} & \begin{minipage}[b]{\linewidth}\raggedright
Código
\end{minipage} & \begin{minipage}[b]{\linewidth}\raggedright
Resultado
\end{minipage} \\
\midrule\noalign{}
\endhead
\texttt{results} & \texttt{markup} & Valor por defecto. Nos muestra los
resultados en el documento final línea a línea, encabezados por
\texttt{\#\#} \\
\texttt{results} & \texttt{hide} & No se nos muestra el resultado en el
documento final \\
\texttt{results} & \texttt{asis} & Nos devuelve los resultados línea a
línea de manera literal en el documento final y el programa con el que
se abre el documento final los interpreta como texto y formatea
adecuadamente \\
\texttt{results} & \texttt{hold} & Miestra todos los resultados al final
del bloque de código \\
\bottomrule\noalign{}
\end{longtable}
\end{frame}

\begin{frame}[fragile]{Parámetros de los chunks}
\phantomsection\label{paruxe1metros-de-los-chunks-2}
```\{r, echo=TRUE, results = ``markup''\}

\texttt{sec\ =\ 10:20}

\texttt{sec}

\texttt{cumsum(sec)}

\begin{Shaded}
\begin{Highlighting}[]

\NormalTok{\textbackslash{}n}


\NormalTok{\textasciigrave{}\textasciigrave{}\textasciigrave{} r}
\NormalTok{sec = 10:20}
\NormalTok{sec}
\end{Highlighting}
\end{Shaded}

\begin{verbatim}
##  [1] 10 11 12 13 14 15 16 17 18 19 20
\end{verbatim}

\begin{Shaded}
\begin{Highlighting}[]
\FunctionTok{cumsum}\NormalTok{(sec)}
\end{Highlighting}
\end{Shaded}

\begin{verbatim}
##  [1]  10  21  33  46  60  75  91 108 126 145 165
\end{verbatim}
\end{frame}

\begin{frame}[fragile]{Parámetros de los chunks}
\phantomsection\label{paruxe1metros-de-los-chunks-3}
```\{r, echo=TRUE, results = ``hide''\}

\texttt{sec\ =\ 10:20}

\texttt{sec}

\texttt{cumsum(sec)}

\begin{Shaded}
\begin{Highlighting}[]

\NormalTok{\textbackslash{}n}


\NormalTok{\textasciigrave{}\textasciigrave{}\textasciigrave{} r}
\NormalTok{sec = 10:20}
\NormalTok{sec}
\NormalTok{cumsum(sec)}
\end{Highlighting}
\end{Shaded}
\end{frame}

\begin{frame}[fragile]{Parámetros de los chunks}
\phantomsection\label{paruxe1metros-de-los-chunks-4}
```\{r, echo=TRUE, results = ``asis''\}

\texttt{sec\ =\ 10:20}

\texttt{sec}

\texttt{cumsum(sec)}

\begin{Shaded}
\begin{Highlighting}[]

\NormalTok{\textbackslash{}n}


\NormalTok{\textasciigrave{}\textasciigrave{}\textasciigrave{} r}
\NormalTok{sec = 10:20}
\NormalTok{sec}
\end{Highlighting}
\end{Shaded}

{[}1{]} 10 11 12 13 14 15 16 17 18 19 20

\begin{Shaded}
\begin{Highlighting}[]
\FunctionTok{cumsum}\NormalTok{(sec)}
\end{Highlighting}
\end{Shaded}

{[}1{]} 10 21 33 46 60 75 91 108 126 145 165
\end{frame}

\begin{frame}[fragile]{Parámetros de los chunks}
\phantomsection\label{paruxe1metros-de-los-chunks-5}
```\{r, echo=TRUE, results = ``hold''\}

\texttt{sec\ =\ 10:20}

\texttt{sec}

\texttt{cumsum(sec)}

\begin{Shaded}
\begin{Highlighting}[]

\NormalTok{\textbackslash{}n}


\NormalTok{\textasciigrave{}\textasciigrave{}\textasciigrave{} r}
\NormalTok{sec = 10:20}
\NormalTok{sec}
\NormalTok{cumsum(sec)}
\end{Highlighting}
\end{Shaded}

\begin{verbatim}
##  [1] 10 11 12 13 14 15 16 17 18 19 20
##  [1]  10  21  33  46  60  75  91 108 126 145 165
\end{verbatim}
\end{frame}

\section{Los chunks en modo línea}\label{los-chunks-en-modo-luxednea}

\begin{frame}[fragile]{Los chunks en modo línea}
\phantomsection\label{los-chunks-en-modo-luxednea-1}
Con lo explicado hasta ahora, solamente hemos generado resultados en la
línea aparte

Para introducir una parte de código dentro de un párrafo y que se
ejecute al comilarse el documento mostrando así el resultado final, hay
que hacerlo utilizando \texttt{\textasciigrave{}r\ ...\textasciigrave{}}

\textbf{Ejemplo}

La raíz cuadrada de 64 es
\texttt{\textasciigrave{}r\ sqrt(64)\textasciigrave{}} o, lo que viene
siendo lo mismo,
\(\sqrt{64}=\)\texttt{\textasciigrave{}r\ sqrt(64)\textasciigrave{}}

La raíz quinta de 32 es 8 o, lo que viene siendo lo mismo,
\$\sqrt{64}=\$5.6568542
\end{frame}

\end{document}
